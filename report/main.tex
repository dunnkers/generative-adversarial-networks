\documentclass{article}
\usepackage[english]{babel}
\usepackage{palatino,mathpazo}
\usepackage[utf8]{inputenc}
\setlength\parindent{0pt}
\pagenumbering{gobble}
\usepackage[margin=1.4in]{geometry}

\usepackage[maxnames=20]{biblatex}
\usepackage{csquotes}
\addbibresource{bibliography.bib}
\usepackage[nottoc,numbib]{tocbibind}

% Lame title idea: Faking Van Gogh :')
% pretty nice title 

\title{Making Art with Generative Adversarial Networks}
\author{Loran Knol (s3182541)\\Thijs Havinga (s2922924)\\Elisa Oostwal (s2468255)\\Jeroen Overschie (s2995697)}
\date{\today}

\begin{document}

\maketitle

\section*{Introduction}

% About Generative Adversarial Networks
Generative Adversarial Networks are a relatively new type of technique, in which two networks are simultaneously trained whilst competing against each other. One of the networks is a \textit{generative} network, which generates new 'candidates' from a lower-dimensional space, trying to represent the data distribution of interest. A candidate could, for example, be a picture of a person or some other entity. The other network, however, is of a \textit{discriminative} type, which tries to classify the candidates produced by the generative network as either real or fake. As such, the two networks play a zero-sum game: the generative network tries to ``fool'' the discriminative network, while the discriminative network tries to tell real from synthesized. The generative network is thus stimulated to create ever more convincing candidates, up to a level where it might be impossible to tell real from fake.\\

% Short description of the project / project goal + Datasets / simulator used
In this project, we aim to train such a Generative Adversarial Network ourselves, with the purpose of image generation specifically. As the generation of human faces has been widely studied, we have chosen for a different topic, namely, the generation of paintings. While large datasets of paintings are available (e.g., ~\cite{kaggle_rijksmuseum}), we have opted to restrict ourselves to one artist, as we believe this will give a better chance at producing realistic paintings. For this, we have chosen the Dutch artist Vincent van Gogh, who is known for his unique style. The dataset is taken from~\cite{kaggle_van_gogh} and consists of $2 \times 10^3$ images. The goal then becomes to generate paintings that resemble Van Gogh's in terms of color usage and style.\\
\\
While our idea may seem similar to programs which can copy the style of one picture to another, such as DeepDream~\cite{deepdream} and deepart.io~\cite{deepart}, there is an important difference: in the two aforementioned mentioned programs two input images, usually a painting and a photo, are combined to a new output image using a Convolutional Neural Network, whereas our program will use a Generative Adversarial Network to generate an output image solely based on a collection of images (paintings). The images that are produced will therefore not follow the structure of a specified input image. As such, we do not expect our results to match up to the ones produced by these programs.\\

% Programming environment

For training the network, Keras was chosen as it offers a wide variety of (pre-trained) models.

% Auxiliary Classifier Generative Adversarial Network ?

% Might be useful: https://github.com/eriklindernoren/Keras-GAN

\printbibliography

\end{document}
